% Created 2018-06-25 Mo 13:07
% Intended LaTeX compiler: pdflatex
\documentclass[11pt]{scrartcl}
\usepackage[utf8]{inputenc}
\usepackage[T1]{fontenc}
\usepackage{graphicx}
\usepackage{grffile}
\usepackage{longtable}
\usepackage{wrapfig}
\usepackage{rotating}
\usepackage[normalem]{ulem}
\usepackage{amsmath}
\usepackage{textcomp}
\usepackage{amssymb}
\usepackage{capt-of}
\usepackage{hyperref}
\author{simon}
\date{\today}
\title{}
\hypersetup{
 pdfauthor={Simon Pfreundschuh},
 pdftitle={},
 pdfkeywords={},
 pdfsubject={},
 pdfcreator={Emacs 24.5.1}, 
 pdflang={English}}
\begin{document}

\setlength{\parindent}{0cm}

\section*{Interactive comment on ``A neural network approach to estimate a posteriori distributions of Bayesian retrieval problems'' by Simon Pfreundschuh et al.}

\subsection*{Referee comment:}

P. 15: in the comparisons between QRNN and the BMCI, as the training data or a-prior
get smaller, the BMCI uncertainties need to be increased beyond the sensor noise to
account for a sparse a-priori.  If that was not done, it likely explains the divergence in
the performance for smaller training sets.  That said, finding the uncertainty due to a
sparse a-priori is not at all trivial so it might still be an advantage for the QRNN but
perhaps slightly different than presented. A bit more explanation by the author on this
topic would help the paper. The conclusion mentions this as well.

\subsection*{Author response:}

This is a very valid point that has indeed not been considered in the presented
calculations. However, in particular since there is no formal way of doing this,
it seems that finding suitable ways of handling scarce databases with BMCI would
merit a study of its own. Applying just any ad-hoc solution to increase the
measurement uncertainty is unlikely to do the BMCI method justice, so the authors
judge it to be out of the scope of the study to investigate this further.

To address this in the manuscript, the following paragraph has been added:

\vspace{1em}

\textit{
A possible approach to handling scarce retrieval databases with BMCI is to
artificially increase the assumed measurement uncertainty. This has not been
performed for the BMCI results presented here and may improve the performance of
the method. The difficulty with this approach is that the method formulation
is based on the assumption of a sufficiently large database and thus can,
at least formally, not be handle scarce training data. Finding a suitable way to
increase the measurement uncertainty would thus require either additional
methodological development or invention of an heuristic approach, both of which
are outside the scope of this study.}

\subsection*{Referee comment}

P. 12, line 12:  Maybe I missed it but I don’t think Rectilinear Linear Unit was ever
defined in the paper.

\subsection*{Author response}

The ReLU activation function is now introduced as Rectified Linear Unit the
first time the acronym is used in the text.

\subsection*{Referee comment}
I am quite certain that neither “Gaussianity” (p.3, line 2) nor “overproportionally” (p. 12,
line 20) are real words.

\subsection*{Author response}

The word \emph{Gaussianity} has been replaced. The sentence now reads:

\vspace{1em}
\textit{
Nonetheless, even neglecting the validity of the assumptions of Gaussian
a priori and measurement errors as well as linearity of the forward
model, the method is unsuitable for retrievals that involve complex
radiative processes.
}
\vspace{1em}

Similarly, \emph{overproportionally} (which correctly should have been
overpropotionately) has been replaced by \emph{excessively}.

\subsection*{Referee comment}

p. 4, line 15: There is an extra “from” in front of “directly”

\subsection*{Author comment}

The word \emph{from} has been removed.
\end{document}
